\begin{thebibliography}{}
	\bibitem{fedorov} Теория оптимального эксперимента (планирование регрессионных экспериментов). Федоров В.В., монография, Главная редакция физико-математической литературы изд-ва "Наука", 1971.
	\bibitem{numpy} Stéfan van der Walt, S. Chris Colbert and Gaël Varoquaux. The NumPy Array: A Structure for Efficient Numerical Computation // Computing in Science \& Engineering. -- 2011. Vol. 13. -- P. 22--30.
	\bibitem{scipy} Pauli Virtanen, Ralf Gommers, Travis E. Oliphant, Matt Haberland, Tyler Reddy, David Cournapeau, Evgeni Burovski, Pearu Peterson, Warren Weckesser, Jonathan Bright, Stéfan J. van der Walt, Matthew Brett, Joshua Wilson, K. Jarrod Millman, Nikolay Mayorov, Andrew R. J. Nelson, Eric Jones, Robert Kern, Eric Larson, CJ Carey, İlhan Polat, Yu Feng, Eric W. Moore, Jake VanderPlas, Denis Laxalde, Josef Perktold, Robert Cimrman, Ian Henriksen, E.A. Quintero, Charles R Harris, Anne M. Archibald, Antônio H. Ribeiro, Fabian Pedregosa, Paul van Mulbregt, and SciPy 1.0 Contributors.  SciPy 1.0: Fundamental Algorithms for Scientific Computing in Python // Nature Methods, in press. -- 2020.
	\bibitem{kirlitsa2017} В.П. Кирлица -- "Точные  D-оптимальные планы экспериментов для линейной множественной регрессии с неравноточными наблюдениями" // Журнал Белорусского государственного университета. Математика, информатика, 2017(3), с. 53-59.
	\bibitem{kirlitsa2019} В.П. Кирлица -- "Построение  D-оптимальных  планов экспериментов для линейной множественной регрессии с неравноточными наблюдениями" // Журнал Белорусского государственного университета. Математика, информатика, 2019(2), с. 27-33.
	\bibitem{aivazian} Айвазян С. А. Прикладная статистика. Основы эконометрики. Том 2. — М.: Юнити-Дана, 2001. — 432 с. — ISBN 5-238-00305-6.
	\bibitem{sympy} Meurer A, Smith CP, Paprocki M, Čertík O, Kirpichev SB, Rocklin M, Kumar A, Ivanov S, Moore JK, Singh S, Rathnayake T, Vig S, Granger BE, Muller RP, Bonazzi F, Gupta H, Vats S, Johansson F, Pedregosa F, Curry MJ, Terrel AR, Roučka Š, Saboo A, Fernando I, Kulal S, Cimrman R, Scopatz A. SymPy: symbolic computing in Python. //  PeerJ Computer Science. -- 2017.  Vol. 3. -- 3:e103.
	\bibitem{scikit-learn} Fabian Pedregosa, Gaël Varoquaux, Alexandre Gramfort, Vincent Michel, Bertrand Thirion, Olivier Grisel, Mathieu Blondel, Peter Prettenhofer, Ron Weiss, Vincent Dubourg, Jake Vanderplas, Alexandre Passos, David Cournapeau, Matthieu Brucher, Matthieu Perrot, Édouard Duchesnay. Scikit-learn: Machine Learning in Python // Journal of Machine Learning Research. -- 2011. Vol. 12. -- P.  2825--2830.
\end{thebibliography}

% {\small \bibliography{diploma-bibliography}}