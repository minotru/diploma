Математической моделью эксперимента является
\begin{equation}
y = f(x, \theta)  
\end{equation}
где $y$ - вектор наблюдаемых значений;\\
$x$ - вектор факторов или контролируемых параметров;\\
$\theta$ - неизвестные параметры;\\
$f(x, \theta)$ - некоторая неизвестная функция.

Модель в случае линейной регрессии имеет вид
\begin{equation}
y = \theta_1 f_1(x) +...+ \theta_m f_m(x) = \theta'f(x)
\end{equation}
где $y$ -  вектор наблюдаемых значений;\\
$(x_1,...,x_m)=x$ - вектор факторов или контролируемых параметров;\\
$(\theta_1,...,\theta_m)=\theta$ - неизвестные параметры;\\
$(f_1(x),...,f_m(x))=f(x)$ - известные функции.\\

Пусть было проведено $n$ наблюдений
\begin{equation}
y_i = \theta' f(x^{(i)}) + \varepsilon_i, i = \overline{1, n}, n \ge m
\end{equation}
где $\varepsilon_i$ - ошибки наблюдений:
$$E\{\varepsilon_i\} = 0$$
$$E\{\varepsilon_i \varepsilon_j\} = 0, i \ne j$$
$$D\{\varepsilon_i\}=\sigma_i ^2$$

Пусть $\hat{\theta} =T(x^{(1)},..,x^{(n)}) $ - некоторая оценка параметра $\theta$.\\


\newtheorem{theorem}{Теорема}

\begin{theorem}[о наилучшей линейной оценке неизвестного параметра]
	\begin{equation}
	\hat{\theta} = M^{-1} Y,
	\end{equation}
	где 
	$$M = \sum_{i=1}^{n} \frac{1}{\sigma_i^2}f(x^{i})f'(x^{i}), |M| \ne 0 \text{ - информационная матрица Фишера};$$
	$$Y = \sum_{i=1}^{n}\frac{1}{\sigma_i^2} y_i f(x^{i}).$$
	И дисперсионная матрица оценок $\hat \theta$ равна
	\begin{equation}D(\hat \theta)  = M^{-1}\end{equation}.
\end{theorem}

Существуют различные критерии оптимальности плана эксперимента. В данной работе будут рассмотрены $D$-оптимальные планы.
\newtheorem{definition}{Определение}
\begin{definition}
	План эксперимента называется $D$-оптимальным, когда
	\begin{equation}\label{def:d-opt}
	|\det D\{\hat \theta\} | \rightarrow \min_{x \in X}
	\end{equation}
\end{definition}
Но
\begin{equation}
|\det {D\{\hat \theta\}}| = \frac 1 {|\det M(x)|} \implies (\ref{def:d-opt}) \iff |det M(x)| \rightarrow \max_{x \in X}
\end{equation}
Таким образом задача построение оптимального плана свелась к максимизации информационной матрицы.

Обозначим
\begin{gather}
X = 
\begin{bmatrix}
x_{11}	&	\dots	&	x_{1m}\\
\dots	&	\dots	&	\dots\\
x_{n1}	&	\dots	&	x_{nm}
\end{bmatrix} \text{ - матрица плана эксперимента} \\
\varepsilon = (\varepsilon_1,.., \varepsilon_n,) \in R^n, E\{\varepsilon\} = 0, E\{\varepsilon \varepsilon'\}=I_n
\end{gather}

Тогда задача (1) в матричной форме:
\begin{equation}
y = X \theta + \varepsilon
\end{equation}

Также без ограничения общности для каждого набора $x$ можем считать, что  $x_k \in [-1, 1]$, так как линейным преобразованием отрезок $x_k \in [a_k, b_k]$ можно перевести в $[-1, 1]$:
$$z_k = \frac{x_k - (a_k + b_k)/2}{(b_k - a_k)/2}, k = \overline{1, m}$$  

\begin{definition}
	Совокупность величин
	\begin{gather}
	x_1, x_2, \dots, x_n;\\
	r_1, r_2, \dots, r_n;\\
	\sum_{i=1}^n r_i = N
	\end{gather}
	называется планом (эксперимента) $\varepsilon(N)$. Совокупность точек  $x_1, x_2, \dots, x_n$ называется спектром плана $\varepsilon(N)$.
\end{definition}

\begin{definition}
	Нормированным планом $\varepsilon(N)$ называется совокупность величин
	\begin{gather}
	x_1, x_2, \dots, x_n;\\
	p_1, p_2, \dots, p_n;\\
	\sum_{i = 1}^n p_i = 1, p_i = \frac {r_i} {N}.
	\end{gather}
\end{definition}

В условиях, когда  $N$ настолько велико, что функцию потерь можно рассматривать как непрерывную по $N$, вводится понятие непрерывного нормированного плана.
\begin{definition}
	Непрерывным нормированным планом называется совокупность величин
	\begin{gather}
	x_1, x_2, \dots, x_n;\\
	\rho_1, \rho_2, \dots, \rho_n;\\
	\sum_{i = 1}^n \rho_i = 1, \rho_i \in (0, 1).
	\end{gather}
\end{definition}
