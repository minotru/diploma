Дипломная работа: \total{page}\ страниц, \totalfigures{}~иллюстраций, \totaltables{}\ таблиц,
\total{citenum}\ источников, 5 приложений.

\vspace{\baselineskip}

ТОЧНЫЕ $D$-ОПТИМАЛЬНЫЕ ПЛАНЫ ЭКСПЕРИМЕНТОВ, ЛИНЕЙНАЯ МНОЖЕСТВЕННАЯ РЕГРЕССИЯ, НЕРАВНОТОЧНЫЕ НАБЛЮДЕНИЯ.

\vspace{\baselineskip}

Объект исследования – неравноточные наблюдения. Цель работы – разработать методы построения $D$-оптимальных планов экспериментов для неравноточных наблюдений.

Методы исследования – методы теории вероятностей, математической
статистики, теория оптимального эксперимента.

Результатами являются построенные $D$-оптимальные планы для неравноточных наблюдений с 2 и 3 независимыми переменными, с линейным изменением.

Областью применения являются планирование научных и производственных экспериментов.
