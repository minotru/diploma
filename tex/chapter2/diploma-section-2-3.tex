В пункте \ref{chapter2/diploma-section-2-1} была рассмотрена модель линейной регрессии \eqref{model:start}-\eqref{model:end} с тремя параметрами и двумя независимыми переменными.
Обобщим данную модель, вводя третью независимую переменную $x_0$:
\begin{gather} \label{model-3x:start}
y_i = \theta_0 x_{i0} + \theta_1 x_{i1} + \theta_2 x_{i2} + \varepsilon(x^{(i)}), i = \overline{1, n}, n \ge 3 \\
E\{ \varepsilon(x^{(i)}) \} = 0, E\{ \varepsilon^{(i)}\ \varepsilon^{(j)} \} = 0, i \ne j \\
D\{ \varepsilon(x^{(i)}) \} = d(x_{i0}, x_{i1}, x_{i2}) > 0, \\
d(x_0, x_1, x_2) \ge \frac{\sigma^2}{3}(x_0^2 + x_1^2 + x_2^2) \label{model-3x:end},
-1 \le x_{ij} \le 1, j = \overline{0, 2}, i = \overline{1, n},
\end{gather}
где $n$ - число наблюдений, $x_{ij}$ - независимые переменные, $x^{(i)} = (x_{i0}, x_{i1}, x_{i2})^T$ - вектор наблюдений, $\varepsilon(x^{(i)})$ - случайные некоррелированные ошибки.

Таким образом все наблюдения находятся внутри куба c центром в точке $(0, 0, 0)$ и стороной 2.
Занумеруем вершины куба:
\begin{align*}
x^{(1)} = (1, 1, 1), && x^{(2)} = (1, -1, 1), && x^{(3)} = (1, -1, -1), && x^{(4)} = (1, 1, -1),\\
x^{(5)} = (-1, 1, 1), && x^{(6)} = (-1, -1, 1), && x^{(7)} = (-1, -1, -1), && x^{(8)} = (-1, 1, -1).
\end{align*}
Точки $x^{(1)}-x^{(4)}$ лежат на верхней грани куба и обозначены так же, как в модели \eqref{model:start}-\eqref{model:end}.\\
Также обозначим значение функции дисперсии в вершинах куба:
\begin{equation}
d_i = d(x^{(i)}), i = \overline{1, 8}.
\end{equation}

В теореме \ref{main-theorem} было показано, что для модели с двумя независимыми переменными \eqref{model:start}-\eqref{model:end} можно построить трёхточеный $D$-оптимальный план.  
Попробуем построить трёхточечный $D$-оптимальный план для модели с тремя независимыми переменными  \eqref{model-3x:start}-\eqref{model-3x:end}.
\begin{theorem}.\label{theorem:model-3x-D-opt}
	Для модели неравноточных наблюдений \eqref{model-3x:start}-\eqref{model-3x:end} следующие трехточечные планы являются непрерывными и $D$-оптимальными.\\
	План 
	\begin{equation} \label{theorem:model-3x-D-opt:plan-1}
	\varepsilon_1^{0} = \left \{ 
	\underset{\frac 1 3} {x^{(1)}},
	\underset{\frac 1 3} {x^{(2)}},
	\underset{\frac 1 3} {x^{(3)}}
	\right \},
	\end{equation}
	если для дисперсии наблюдений выполняется:
	\begin{multline}\label{theorem:model-3x-D-opt:plan-1-d}
	d(x_0, x_1, x_2) \ge \frac{d_{2} x_{1} x_{2}}{2} + x_{0}^{2} \left(\frac{d_{1}}{4} + \frac{d_{3}}{4}\right) + x_{0} \left(\frac{d_{1} x_{1}}{2} - \frac{d_{3} x_{2}}{2}\right) + \\ + x_{1}^{2} \left(\frac{d_{1}}{4} + \frac{d_{2}}{4}\right) + x_{2}^{2} \left(\frac{d_{2}}{4} + \frac{d_{3}}{4}\right).
	\end{multline}
	План 
	\begin{equation} \label{theorem:model-3x-D-opt:plan-2}
	\varepsilon_2^{0} = \left \{ 
	\underset{\frac 1 3} {x^{(1)}},
	\underset{\frac 1 3} {x^{(2)}},
	\underset{\frac 1 3} {x^{(4)}}
	\right \},
	\end{equation}
	если для дисперсии наблюдений выполняется:
	\begin{multline}\label{theorem:model-3x-D-opt:plan-2-d}
	d(x_0, x_1, x_2) \ge \frac{d_{1} x_{1} x_{2}}{2} + x_{0}^{2} \left(\frac{d_{2}}{4} + \frac{d_{4}}{4}\right) + x_{0} \left(- \frac{d_{2} x_{1}}{2} - \frac{d_{4} x_{2}}{2}\right) + \\ + x_{1}^{2} \left(\frac{d_{1}}{4} + \frac{d_{2}}{4}\right) + x_{2}^{2} \left(\frac{d_{1}}{4} + \frac{d_{4}}{4}\right).
	\end{multline}
	План 
	\begin{equation} \label{theorem:model-3x-D-opt:plan-3}
	\varepsilon_3^{0} = \left \{ 
	\underset{\frac 1 3} {x^{(1)}},
	\underset{\frac 1 3} {x^{(3)}},
	\underset{\frac 1 3} {x^{(4)}}
	\right \},
	\end{equation}
	если для дисперсии наблюдений выполняется:
	\begin{multline}\label{theorem:model-3x-D-opt:plan-3-d}
	d(x_0, x_1, x_2) \ge \frac{d_{4} x_{1} x_{2}}{2} + x_{0}^{2} \left(\frac{d_{1}}{4} + \frac{d_{3}}{4}\right) + x_{0} \left(\frac{d_{1} x_{2}}{2} - \frac{d_{3} x_{1}}{2}\right) + \\ + x_{1}^{2} \left(\frac{d_{3}}{4} + \frac{d_{4}}{4}\right) + x_{2}^{2} \left(\frac{d_{1}}{4} + \frac{d_{4}}{4}\right).
	\end{multline}
	План 
	\begin{equation} \label{theorem:model-3x-D-opt:plan-4}
	\varepsilon_4^{0} = \left \{ 
	\underset{\frac 1 3} {x^{(2)}},
	\underset{\frac 1 3} {x^{(3)}},
	\underset{\frac 1 3} {x^{(4)}}
	\right \},
	\end{equation}
	если для дисперсии наблюдений выполняется:
	\begin{multline}\label{theorem:model-3x-D-opt:plan-4-d}
	d(x_0, x_1, x_2) \ge \frac{d_{3} x_{1} x_{2}}{2} + x_{0}^{2} \left(\frac{d_{2}}{4} + \frac{d_{4}}{4}\right) + x_{0} \left(\frac{d_{2} x_{2}}{2} + \frac{d_{4} x_{1}}{2}\right) + \\ + x_{1}^{2} \left(\frac{d_{3}}{4} + \frac{d_{4}}{4}\right) + x_{2}^{2} \left(\frac{d_{2}}{4} + \frac{d_{3}}{4}\right).
	\end{multline}		
\end{theorem}

\textbf{Доказательство}. Доказательство данной теоремы производится аналогично доказательству теоремы \ref{main-theorem}. Для плана в точках $x^{(i)}, x^{(j)}, x^{(k)}$ запишем критерий эквавалентности Кифера-Вольфовица \eqref{theorem:model-3x-D-opt:KF-ineq}:
\begin{equation}\label{theorem:model-3x-D-opt:KF-ineq}
\frac{1}{d(x_0, x_1, x_2)} f'(x) M^{-1}(\varepsilon^0) f(x) \le 3,
\end{equation}
выразим $d(x_0, x_1, x_2)$ и покажем, что $d(x_0, x_1, x_2)$ равно заданным дисперсиям $d_i, d_j, d_k$ в соответствующих точках.

Ввиду объёмности и однотипности выкладок, описанная схема доказательства была реализована программно при помощи пакета компьютерной алгебры SymPy, написанном на языке Python. Исходный код программы приведён в приложении \ref{appendix:3x_cube}.

Результат работы программы:
\lstinputlisting[numbers=none]{listings/3x_cube_output.txt}
Вывод показывает, что неравенства обращаются в равенства в точках спектра плана, что и требовалось доказать. \textbf{Теорема доказана}.\\

\textbf{Следствие}.
В теореме \ref{theorem:model-3x-D-opt} были рассмотрены планы, точки которых находятся на верхней грани куба. Куб имеет 6 граней, причём на каждой из них можно построить аналогичные планы. Данный факт следует из того, что точки планов на других гранях эквиваленты точкам одного из планов на верхней грани с точностью до переименования независимых переменных $x_0, x_1, x_2$ и домножения на $-1$. Таким образом доказано существование $24$-х планов для модели \eqref{model-3x:start}-\eqref{model-3x:end}.	

