\appendix

\section{Численная проверка оптимальности плана методом перебора точек единичного квадрата}
\label{appendix:check_optimal_in_points}
\lstinputlisting[
	language=Python,
	label={listings:check_optimal_in_points},
	caption={Численная проверка оптимальности плана методом перебора точек единичного квадрата},
	captionpos=bc,
]
{listings/check_optimal_in_points.py}

\cleardoublepage


\section{Символьная проверка оптимальности плана}
\label{appendix:symbol_check_optimal_in_points}
\lstinputlisting[
	language=Matlab,
	label={listings:symbol_check_optimal_in_points},
	caption={Символьная проверка оптимальности плана},
]{listings/symbol_check_optimality.txt}

\cleardoublepage

\section{Размещение наблюдений в точках спектра плана}
\label{appendix:check_ns}
\lstinputlisting[
	language=Python,
	label={listings:check_ns},
	caption={Размещение наблюдений в точках спектра плана}
]{listings/check_ns.py}

\cleardoublepage

\section{Символьная проверка D-оптимальности плана с тремя переменными}
\label{appendix:3x_cube}
\lstinputlisting[
	language=Python,
	label={listings:3x_cube},
	caption={Символьная проверка D-оптимальности плана с тремя переменными}
]{listings/3x_cube.py}

\cleardoublepage

\section{Сравнение качества оценок точного D-оптимального и случайного плана}
\label{appendix:heteroscedastic_experiment}
\lstinputlisting[
	language=Python,
	label={listings:heteroscedastic_experiment},
	caption={Символьная проверка D-оптимальности плана с тремя переменными}
]{listings/heteroscedastic_experiment.py}
