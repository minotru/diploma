Эксперимент является неотъемлемой частью научной и практической деятельности, но притом сложной и затратной. Так, произведя новый товар, необходимо узнать его характеристики (например, прочность), с целью чего производится серия экспериментов. Экспериментатор заинтересован в получении наилучших результатов при наименьших затратах, и важную роль в достижении этой цели играет план эксперимента.\\

В данной работе рассмотрен вопрос  построения оптимального плана эксперимента для случая, когда неизвестная зависимость приближается линейной регрессией, а наблюдения гетероскедастичны (неравноточны). Уточним, что для равноточных наблюдений существует развитая теория, а в данной работе показано обобщение теории на случай неравноточных наблюдений.